\documentclass[onecolumn,11pt,nocopyrightspace]{sigplanconf}
\usepackage[T1]{fontenc}
\usepackage[utf8]{inputenc}
\usepackage[bookmarks=true,bookmarksopen=true,colorlinks=true,%
            linkcolor=blue,citecolor=blue,urlcolor=blue]{hyperref}
\usepackage{listings}
% Configuring listings for OCaml.

% Comments in blue.
\newcommand{\ocamlcommentstyle}{\mycolor{blue}\itshape}

\lstdefinelanguage{ocaml}[Objective]{Caml}{
  % Fix errors in the default definition of ocaml.
  deletekeywords={closed,ref},
  morekeywords={initializer},
  % General settings.
  flexiblecolumns=false,
  showstringspaces=false,
  framesep=5pt,
  commentstyle=\ocamlcommentstyle,
  % By default, we use a small font.
  basicstyle=\tt\small,
  numberstyle=\footnotesize,
  % LaTeX escape.
  escapeinside={$}{$},
}

% An abbreviation for \lstinline, with a normal font size.
% To be used in the text of the paper.
\def\oc{\lstinline[language=ocaml,basicstyle=\tt,flexiblecolumns=true]}

\lstset{language=ocaml}
\usepackage{xcolor}
\usepackage{mdframed}
\usepackage{xspace}
% ------------------------------------------------------------------------------
% Including a source code file foo.ml:
%   \orig{foo}

\mdfsetup{%
  linewidth=0pt,%
  skipabove=\baselineskip,%
  skipbelow=.4\baselineskip,%
  innertopmargin=0pt,%
  innerbottommargin=0pt,%
}

\newenvironment{origenv}{%
  \begin{mdframed}[backgroundcolor=gray!10]%
}{%
  \end{mdframed}%
}

\newcommand{\orig}[1]{%
\begin{origenv}
\lstinputlisting{#1.ml}
\end{origenv}
}

\newcommand{\origfirstline}[2]{%
\begin{origenv}
\lstinputlisting[firstline=#2]{#1.ml}
\end{origenv}
}

%% \newcommand{\codefollowupgeneral}[1]{%
%% \begin{center}
%% (This code fragment follows up on #1.)
%% \end{center}
%% \vspace{-\baselineskip}
%% }

\newcommand{\codefollowupgeneral}[1]{%
\marginnote{\footnotesize This follows #1.}[5mm]%
}

\newcommand{\codefollowup}[1]{%
  \codefollowupgeneral{\fref{fig:#1}}%
}

% ------------------------------------------------------------------------------
% Including the corresponding auto-generated code:
%   \processed{foo}

\newenvironment{processedenv}{%
  \begin{mdframed}[backgroundcolor=yellow!40]%
}{%
  \end{mdframed}%
}

\newcommand{\processed}[1]{%
\begin{processedenv}
\lstinputlisting{#1.processed.ml}
\end{processedenv}
}

% ------------------------------------------------------------------------------
% Various macros.

\newcommand{\fref}[1]{Figure~\ref{#1}}
\newcommand{\sref}[1]{\S\ref{#1}}
\newcommand{\email}[1]{\href{mailto:#1}{\texttt{#1}}}

% ------------------------------------------------------------------------------
% Abbreviations.

\newcommand{\merlin}{\href{https://github.com/ocaml/merlin}{Merlin}\xspace}
\newcommand{\ocamlbuild}{\href{https://github.com/ocaml/ocamlbuild/blob/master/manual/manual.adoc}{\texttt{ocamlbuild}}\xspace}
\newcommand{\ocamlfind}{\texttt{ocamlfind}\xspace}
\newcommand{\opam}{\href{https://opam.ocaml.org/}{\texttt{opam}}\xspace}
\newcommand{\ppxderiving}{\href{https://github.com/whitequark/ppx_deriving}{\texttt{ppx\_deriving}}\xspace}
\newcommand{\ppximport}{\href{https://github.com/whitequark/ppx_import}{\texttt{ppx\_import}}\xspace}
\newcommand{\dune}{\href{https://github.com/ocaml/dune}{Dune}\xspace}

\newcommand{\hashconsRepoURL}{https://github.com/backtracking/ocaml-hashcons}
\newcommand{\hashcons}{\href{\hashconsRepoURL}{\texttt{hashcons}}\xspace}
\newcommand{\hashconsRepoFile}[1]{\href{\hashconsRepoURL/blob/master/#1}{\texttt{#1}}}

\newcommand{\repoURL}{https://gitlab.inria.fr/fpottier/visitors}
\newcommand{\repoFile}[1]{\href{\repoURL/blob/master/#1}{\texttt{#1}}}
% \newcommand{\srcFile}[1]{\href{\repoURL/blob/master/src/#1}{\texttt{#1}}}
\newcommand{\runtimeFile}[1]{\href{\repoURL/blob/master/runtime/#1}{\texttt{#1}}}

\newcommand{\iter}{\texttt{iter}\xspace}
\newcommand{\map}{\texttt{map}\xspace}
\newcommand{\mapendo}{\texttt{endo}\xspace}
\newcommand{\reduce}{\texttt{reduce}\xspace}
\newcommand{\mapreduce}{\texttt{mapreduce}\xspace}
\newcommand{\fold}{\texttt{fold}\xspace}
\newcommand{\itertwo}{\texttt{iter2}\xspace}
\newcommand{\maptwo}{\texttt{map2}\xspace}
\newcommand{\reducetwo}{\texttt{reduce2}\xspace}
\newcommand{\mapreducetwo}{\texttt{mapreduce2}\xspace}
\newcommand{\foldtwo}{\texttt{fold2}\xspace}
\newcommand{\unit}{\texttt{unit}\xspace}
\newcommand{\visitors}{\texttt{visitors}\xspace}
\newcommand{\derivingvisitors}{\texttt{[@@deriving visitors \{ ...\ \}]}\xspace}
\newcommand{\runtime}[1]{\oc|VisitorsRuntime.#1|}

\newcommand{\tyconvisitor}[1]{\texttt{visit\_#1}}
\newcommand{\dataconvisitor}[1]{\texttt{visit\_#1}}
\newcommand{\tyconfail}[1]{\texttt{fail\_#1}}
\newcommand{\dataconascendingmethod}[1]{\texttt{build\_#1}}
\newcommand{\tyconascendingmethod}[1]{\texttt{build\_#1}}

\newcommand{\refconvention}{\sref{sec:intro:parameterized:poly}, \sref{sec:intro:nonlocal},
\fref{fig:convention}}

% Options.
\newcommand{\ancestors}{\texttt{ancestors}\xspace}
\newcommand{\concrete}{\texttt{concrete}\xspace}
\newcommand{\irregular}{\texttt{irregular}\xspace}
\newcommand{\name}{\texttt{name}\xspace}
\newcommand{\public}{\texttt{public}\xspace}
\newcommand{\variety}{\texttt{variety}\xspace}
\newcommand{\data}{\texttt{data}\xspace}
\newcommand{\nude}{\texttt{nude}\xspace}
\newcommand{\polymorphic}{\texttt{polymorphic}\xspace}
\newcommand{\visitprefix}{\texttt{visit\_prefix}\xspace}
\newcommand{\buildprefix}{\texttt{build\_prefix}\xspace}
\newcommand{\failprefix}{\texttt{fail\_prefix}\xspace}


% ------------------------------------------------------------------------------
% Headings.

\title{Visitors}

\begin{document}
\authorinfo{François Pottier}{Inria Paris}{francois.pottier@inria.fr}
\maketitle

% ------------------------------------------------------------------------------

% TEMPORARY
\clearpage
% \tableofcontents
% \clearpage

% ------------------------------------------------------------------------------

\section{Introduction}

\subsection{Setup}

In order to install the \visitors package, an \opam user should issue the
following commands:
\begin{verbatim}
  opam update
  opam install visitors
\end{verbatim}
In order to use the \visitors package, an \ocamlbuild user should add the
following line in her project's \texttt{\_tags} file:
\begin{lstlisting}
  true: package(visitors.ppx), package(visitors.runtime)
\end{lstlisting}
Furthermore, a Merlin user should add the following lines in her project's
\texttt{.merlin} file:
\begin{lstlisting}
  PKG visitors.ppx
  PKG visitors.runtime
\end{lstlisting}

% ------------------------------------------------------------------------------

\subsection{A first example}

\begin{figure}[t]
In an OCaml source file, a type definition can be annotated with
\oc|[@@deriving visitors { ... }]|:
\orig{expr00}
This causes the following code to be (invisibly) generated:
\processed{expr00}
\caption{A visitor of the \iter variety}
\label{fig:expr00}
\end{figure}

Suppose we wish to work with arithmetic expressions built out of integer
literals and binary additions. An algebraic data type \oc|expr| is shown in
the first part of \fref{fig:expr00}. By annotating this type definition with
\oc|[@@deriving visitors { ... }]|, we request the automated generation of
a visitor for expressions. The code for this visitor, which in normal use
remains invisible, is shown in the second part of \fref{fig:expr00}.

The annotation \oc|[@@deriving visitors { ... }]| must carry at least two
parameters. The parameter \name can be an arbitrary string; it is used as the
name of the generated class. The parameter \variety must be one of \iter,
\map, \itertwo, \maptwo. It indicates which variety of visitor must be
generated.

A visitor takes the form of an OCaml class, whose methods are named after the
types and data constructors that appear in the type definition. In
\fref{fig:expr00}, for instance, the method \tyconvisitor{expr} is named after
the type \oc|expr|, while the methods \dataconvisitor{EConst} and
\dataconvisitor{EAdd} are named after the data constructors \oc|EConst| and
\oc|EAdd|.

A visitor of the \iter variety traverses a data structure. It returns no
result: its methods have return type \unit. The default implementations of
these methods do nothing. In \fref{fig:expr00}, for instance, the method
\tyconvisitor{expr} inspects its argument \oc|this_expr| and recursively
invokes either \dataconvisitor{EConst} or \dataconvisitor{EAdd}, as
appropriate. The method \dataconvisitor{EConst} does nothing. (More precisely,
it calls the function \oc|VisitorsRuntime.Int.iter|, which itself does
nothing.) The method \dataconvisitor{EAdd} performs two recursive calls to
\tyconvisitor{expr}, which does nothing, so \dataconvisitor{EAdd} itself does
nothing.

Every method is parameterized with an environment \oc|env|, which is carried
down into every recursive call and is otherwise unused.
% The type of this environment is therefore a priori undetermined.

% ------------------------------------------------------------------------------

\begin{figure}[t]
\orig{expr01}
\vspace{-\baselineskip}
\processed{expr01}
\caption{A visitor of the \map variety}
\label{fig:expr01}
\end{figure}

% ------------------------------------------------------------------------------

\begin{figure}[t]
\orig{expr02}
\vspace{-\baselineskip}
\processed{expr02}
\caption{An visitor of the \itertwo variety}
\label{fig:expr02}
\end{figure}

% ------------------------------------------------------------------------------

\begin{figure}[t]
\orig{expr03}
\vspace{-\baselineskip}
\processed{expr03}
\caption{A visitor of the \maptwo variety}
\label{fig:expr03}
\end{figure}

% ------------------------------------------------------------------------------

\end{document}
